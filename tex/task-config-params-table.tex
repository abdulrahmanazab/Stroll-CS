%!TEX root = ../main.tex

\begin{scriptsize}
\begin{table}[htbp]
\caption{Task parameters (files in the \cmd{config} directory)}
\label{tab:config}
\centering
\begin{tabular}{|c|p{6 cm}|} \hline 

\textsc{Config} & \textsc{Description}\\ \hline\hline

\textbf{exec} & Executable file name.\\ \hline

\textbf{rt} & Runtime environment needed to run the executable, e.g.
JVM, R, or CLR.\\ \hline

\textbf{wait} & Value \textbf{true} causes the \cmd{submit} command to hold 
until task termination. \\ \hline

\textbf{subtasks} & Number of parallel subtasks to be executed for batch 
tasks. \\ \hline

\textbf{in} & Comma separated list of input files. For batch tasks, the input
file names should include a zero based index, e.g. in1,in2,..,in20 for 20
subtasks. \\ \hline

\textbf{out} & Comma separated list of output files. \\ \hline

\textbf{args} & Comma separated list of task arguments. \\ \hline

\textbf{req} & Resource requirements, e.g. memory and load average. The
format is a similar to Condor's ClassAd~\cite{condor-submit}, only more
compact: \cmd{LA0.1 \&\& M512} corresponds to \cmd{LoadAvg<=0.1 \&\&
Memory>=512}. \\ \hline

\rowcolor{lightgray} \textbf{parents} & A comma separated list of the parent tasks. This configuration parameter is used for configuring parent-child relations in \textbf{composite} tasks.\\\hline

\end{tabular}
\end{table}
\end{scriptsize}

