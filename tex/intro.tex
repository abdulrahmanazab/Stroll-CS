%!TEX root = ../main.tex

\section{Introduction}
\label{sec:intro}

Sensitive data is defined as information that is protected against unwarranted disclosure. Sensitive personal data is any information which is related to an identified or identifiable natural person. Research often involves the use of personal data as a basis for the scientific analysis. However, a particular challenge is to use these data resources without violating privacy. And for that we need secure digital infrastructures, compliant with both national and international regulations, GDPR~\cite{gdpr}.
TSD (Services for Sensitive Data)~\cite{tsd} is a system and e-infrastructure for providing storage and processing of sensitive research data. This infrastructure provides a large storage space and  High Performance Computing (HPC) for data processing. TSD e-infrastructure is divided from the inside into project areas; each is associated with a specific research project. Each project owns an amount of storage space, and all projects can get access to the HPC facilities for running computationally intensive analyses. TSD is a member of the Nordic collaboration for sensitive data, Tryggve~\cite{tryggve}, together with two other sensitive data e-infrastructures. Tryggve e-infrastructures are secure, but their isolation introduces additional complexity for data import, analysis, and export. This complexity results in more manual work to be carried out by researchers to perform data processing in e.g. TSD, especially in use cases where raw data is generated frequently, e.g. a DNA sequencer located in a hospital is generating sequences, and the results of analyzing these sequences need to be sent back to the hospital. This would impose many manual data import/export steps that are effort and time consuming for researchers.   

\name~\cite{stroll} is a universal interface for seamless deployment of compute jobs to a variety of HPC platforms. \name is based on the ubiquitous \fs interface, and allows task submission, monitoring, and administration through simple \rc and \wc \fs functions or commands. The interface is implemented as a user space \emph{virtual \fs}(VFS), enabling access to it from any application or programming language that can interact with a \fs, and supporting both simple and workkflow jobs~\cite{stroll2}. \name is already deployed in TSD to submit jobs from an internal Galaxy portal to the TSD HPC cluster~\cite{docker-galaxy-stroll}.

This paper presents \name client-server architecture in which the \fs front-end, i.e. \name client (\name-C), is deployed on the machine/device where the raw data is generated while \name server (\name-S) together with the HPC cluster front-end are deployed on another machine. Communication and data transfer between \name-C and \name-S is carried out through SFTP with two-factor authentication enabled. The aim of this design is to enable submitting jobs from outside the secure environment to the HPC cluster inside the secure environment, using \fs commands. \name is currently used in a pilot project for job submission in TSD, and is under testing for implementation at the other Tryggve sites. There is an ongoing effort for building a Nordic federated sensitive data cloud where job migration and meta-scheduling is managed by broker overlay based decentralized scheduling. \name is the front-end component in this architecture.

The paper is organized as follows: \secref{sec:stroll-arch} describes the \name \fs Architecture. \secref{sec:tsd} introduces TSD as an e-infrastructure and a service. \secref{sec:stroll-tsd} describes the \name based job management model for TSD. \secref{sec:performance} describes the performance evaluation of \name based job submission. \secref{sec:stroll-tryggve} describes the Tryggve cloud federation architecture with \name as a front-end. \secref{sec:conclusions} presents conclusions and future work directions.  
